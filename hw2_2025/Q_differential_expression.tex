{\bf [30 points] Differential gene expression analysis}\\

In this exercise we will complete steps to identify differentially expressed genes from the size factor normalized read count data you saved as 'size\_factor\_normalized\_counts.txt', from Q1. We will also learn how to perform multiple hypothesis testing to identify significant differentially expressed genes.

You have been supplied with the numerical sample labels in \textit{labels.txt}. There are two types of samples, 1) cell treated with dexamethasone(case) and 2) cell treated with ethanol (control). The gene names have been provided in \textit{GeneNames.txt} file.

\vspace{20pt}

\fbox{\parbox{0.8\textwidth}{
\textbf{Note: }{Your code for this problem is graded by an autograder. The script should be able to run with command line:\\

\texttt{python de\_genes.py }\\

Include requested plots in the pdf file.}
}
}

\vspace{20pt}

\textbf{Your task}

\vspace{20pt}

\begin{enumerate}

\item (10 points) Dispersion of gene $i$ in condition $C$, $C\in \{Case, Control\}$ can be computed as 
 
$$disp_i^{C}=\frac{std(\hat{k}_{i,C})}{mean(\hat{k}_{i,C})}$$ 

Submit a log2-log2 scatterplot of dispersion vs mean in the size factor counts dataset, for the two conditions (but with all the data on one plot). Use different colors for case and control condition. How would you interpret this plot, and why?

%%%%%%%%%%%%%%%%%%
\begin{solution}
~
\end{solution}
%%%%%%%%%%%%%%%%%%

\item (10 points) Compute the log2 fold change of the genes in the two conditions. Select top $10$ upregulated genes, and top $10$ downregulated genes, and submit a gene expression heatmap for the selected genes. Place both upregulated and dowregulated genes on the same heatmap. Are these the most useful genes to investigate? Why or why not?

%%%%%%%%%%%%%%%%%%
\begin{solution}
~
\end{solution}
%%%%%%%%%%%%%%%%%%

\item (10 points) Determining whether the gene expressions in two conditions are statistically different consists of rejecting the null hypothesis that the two data samples come from distributions with equal means. To do this, we can calculate a p-value for each gene.

However. thresholding P-values to determine what fold changes are more significant than others is not appropriate for this type of data analysis, due to the multiple hypothesis testing problem. When performing a large number of simultaneous tests, the probability of getting a significant result simply due to chance increases with the number of tests. In order to account for multiple testing, perform a correction (or adjustment) of the P-values so that the probability of observing at least one significant result due to chance remains below the desired significance level. 
We can use The Benjamini-Hochberg (BH) adjustment.

The BH is defined as:

Let $p_1\leq ... \leq p_n$ be ordered p-values. Define
$$k = i : p_i \leq \frac{i}{n}\alpha$$
and reject $H_0^1...H_0^k$. $\alpha$ is the false discovery rate(FDR) to be controlled. If no such $i$ exists, then no hypothesis will be rejected. 

\textbf{Implement the BH correction function in }\texttt{de\_genes.py}. The implementation must be your own.\\

\end{enumerate}


